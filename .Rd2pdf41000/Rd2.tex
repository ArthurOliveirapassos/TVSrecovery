\nonstopmode{}
\documentclass[a4paper]{book}
\usepackage[times,inconsolata,hyper]{Rd}
\usepackage{makeidx}
\usepackage[utf8]{inputenc} % @SET ENCODING@
% \usepackage{graphicx} % @USE GRAPHICX@
\makeindex{}
\begin{document}
\chapter*{}
\begin{center}
{\textbf{\huge Package `TVSrecovery'}}
\par\bigskip{\large \today}
\end{center}
\inputencoding{utf8}
\ifthenelse{\boolean{Rd@use@hyper}}{\hypersetup{pdftitle = {TVSrecovery: Prepare and analyze FASTA files to identify telomeric variants}}}{}
\begin{description}
\raggedright{}
\item[Type]\AsIs{Package}
\item[Title]\AsIs{Prepare and analyze FASTA files to identify telomeric variants}
\item[Version]\AsIs{0.1.0}
\item[Author]\AsIs{c(
person(``Arthur'', ``Passos'', email = ``arthur.passos@unesp.br'',
          role = c(``aut'', ``cre''))}
\item[Maintainer]\AsIs{Arthur Passos }\email{arthur.passos@unesp.br}\AsIs{}
\item[Description]\AsIs{Provides tools to identify, extract and visualize telomeric variant sequences (TVSs) from FASTA or FASTQ files. Includes functions for detection, plotting, and generating example datasets.}
\item[License]\AsIs{MIT + file LICENSE}
\item[Encoding]\AsIs{UTF-8}
\item[LazyData]\AsIs{true}
\item[URL]\AsIs{}\url{https://github.com/ArthurOliveirapassos/TVSrecovery}\AsIs{}
\item[BugReports]\AsIs{}\url{https://github.com/ArthurOliveirapassos/TVSrecovery/issues}\AsIs{}
\item[RoxygenNote]\AsIs{7.3.3}
\item[Imports]\AsIs{dplyr,
Biostrings,
ggplot2,
ShortRead,
stats,
tools}
\end{description}
\Rdcontents{\R{} topics documented:}
\inputencoding{utf8}
\HeaderA{detect\_TVS}{Detect TVSs (Telomeric Variant Sequences) in FASTA sequences}{detect.Rul.TVS}
%
\begin{Description}
The `detect\_TVS()` function identifies canonical telomeric repeats and
sequence variations occurring between consecutive repeats in FASTA files.
It automatically processes both the forward strand and the reverse-complement
strand, ensuring that all potential telomeric regions are evaluated.
\end{Description}
%
\begin{Usage}
\begin{verbatim}
detect_TVS(fasta_file, canonical_repeat, min_var_len = 2, max_var_len = 8)
\end{verbatim}
\end{Usage}
%
\begin{Arguments}
\begin{ldescription}
\item[\code{fasta\_file}] Path to the FASTA file containing the sequences.

\item[\code{canonical\_repeat}] Canonical repeat sequence (e.g., `"TTAGGG"`).

\item[\code{min\_var\_len}] Minimum allowed size for a TVS (default: 2).

\item[\code{max\_var\_len}] Maximum allowed size for a TVS (default: 8).
\end{ldescription}
\end{Arguments}
%
\begin{Details}
The algorithm operates in three main steps:
1. Identifies all **canonical** occurrences of the provided telomeric repeat.
2. For each pair of consecutive repeats, extracts the intermediate sequence.
3. If this sequence has between `min\_var\_len` and `max\_var\_len` bases
and is not identical to the canonical repeat, it is recorded as a TVS (variant).

The output combines both canonical repeats and variants, enabling further
visualization through `plot\_TVS()`.
\end{Details}
%
\begin{Value}
A `data.frame` containing canonical repeats and variants, with columns:
\begin{description}

\item[seqnames] Name of the sequence where the repeat/TVS was found
\item[start] Start position
\item[end] End position
\item[width] Length of the repeat/variant
\item[type] Type: `"Canonical"` or `"Variant"`
\item[seq\_match] Matched sequence
\item[variant] TRUE for variants, FALSE for canonical repeats

\end{description}

If no repeats or TVSs are found, returns `NULL`.
\end{Value}
%
\begin{Examples}
\begin{ExampleCode}
## Not run: 
result <- detect_TVS(
  arquivo_fasta = "telomere_hits.fasta",
  repeticao_canon = "TTAGGG",
  min_var_len = 2,
  max_var_len = 8
)

## End(Not run)

\end{ExampleCode}
\end{Examples}
\inputencoding{utf8}
\HeaderA{fastq2fasta}{Convert FASTQ file to FASTA}{fastq2fasta}
%
\begin{Description}
This function reads a FASTQ file and generates a FASTA file with the same
IDs and sequences. The output filename can be defined by the user or
automatically generated from the input filename.
\end{Description}
%
\begin{Usage}
\begin{verbatim}
fastq2fasta(file, output_name = NULL)
\end{verbatim}
\end{Usage}
%
\begin{Arguments}
\begin{ldescription}
\item[\code{file}] Path to the input FASTQ file.

\item[\code{output\_name}] Name of the output FASTA file.
If NULL or empty, it will be automatically generated as `"file\_converted.fasta"`.
\end{ldescription}
\end{Arguments}
%
\begin{Value}
Invisibly returns the name of the generated file.
\end{Value}
%
\begin{Examples}
\begin{ExampleCode}
## Not run: 
fastq2fasta("reads.fastq")
fastq2fasta("reads.fastq", output_name = "data.fasta")

## End(Not run)

\end{ExampleCode}
\end{Examples}
\inputencoding{utf8}
\HeaderA{giveExampleFasta}{Generate an example FASTA file with telomeric and random sequences}{giveExampleFasta}
%
\begin{Description}
This function creates a FASTA file containing a mixed set of sequences:
some of them contain canonical telomeric repeats (optionally with variants),
while the remaining sequences are fully random.
\end{Description}
%
\begin{Usage}
\begin{verbatim}
giveExampleFasta(
  output = "example_sequences.fasta",
  n_seqs = 100,
  telomere_repeat = "TTAGGG",
  min_repeats = 3,
  max_repeats = 8,
  variant_prob = 0.3
)
\end{verbatim}
\end{Usage}
%
\begin{Arguments}
\begin{ldescription}
\item[\code{output}] Name of the output FASTA file.
Default: `"example\_sequences.fasta"`.

\item[\code{n\_seqs}] Total number of sequences to be generated.
Default: 100.

\item[\code{telomere\_repeat}] Canonical telomeric repeat to be used.
Default: `"TTAGGG"`.

\item[\code{min\_repeats}] Minimum number of consecutive repeats per telomeric sequence.

\item[\code{max\_repeats}] Maximum number of consecutive repeats per telomeric sequence.

\item[\code{variant\_prob}] Probability of generating a variant within the repeat block.
\end{ldescription}
\end{Arguments}
%
\begin{Details}
The goal is to provide a synthetic dataset useful for testing, demonstrations,
and examples within the package itself, such as in README chunks or vignettes.


The function randomly generates:

* Between **40 and 60 telomeric sequences** containing canonical repeats
with randomly introduced variants.
* The remaining sequences (`n\_seqs - telomeric`) will be **fully random**.

The function also adds random flanking regions around the telomeric block.

Variants are generated by substituting a single base inside the canonical repeat.
\end{Details}
%
\begin{Value}
A `DNAStringSet` object containing all generated sequences
(also written to the output FASTA file).
\end{Value}
%
\begin{Examples}
\begin{ExampleCode}
## Not run: 
# Generate an example FASTA file in the current directory
example_data <- giveExampleFasta(
  output = "example.fasta",
  n_seqs = 50,
  telomere_repeat = "TTAGGG",
  variant_prob = 0.2
)

# View the first sequences
example_data[1:5]

## End(Not run)

\end{ExampleCode}
\end{Examples}
\inputencoding{utf8}
\HeaderA{plot\_TVS}{Plot telomeric variants and canonical vs variant proportion}{plot.Rul.TVS}
%
\begin{Description}
This function takes the data frame generated by `detect\_TVS()` and produces
two useful plots for visualizing telomeric repeats:
\end{Description}
%
\begin{Usage}
\begin{verbatim}
plot_TVS(variant_df)
\end{verbatim}
\end{Usage}
%
\begin{Arguments}
\begin{ldescription}
\item[\code{variant\_df}] A data frame generated by `detect\_TVS()`, containing
canonical and variant repeats. The object must include the columns:
`seq\_match`, `type`, and `variant`.
\end{ldescription}
\end{Arguments}
%
\begin{Details}
**1. Bar plot** showing the most frequent variants
**2. Pie chart** showing the proportion between canonical and variant repeats
\end{Details}
%
\begin{Value}
A list containing two `ggplot2` objects:
\begin{description}

\item[bar\_plot] Bar plot with recurrent variants (`n > 1`).
\item[pie\_plot] Pie chart showing the proportion of canonical vs variant repeats.

\end{description}


If the data frame is empty, the function returns `NULL` and issues a warning.
\end{Value}
%
\begin{Examples}
\begin{ExampleCode}
## Not run: 
result <- detect_TVS("example.fasta", repeticao_canon = "TTAGGG")
plots <- plot_TVS(result)

# Display plots
plots$bar_plot
plots$pie_plot

## End(Not run)

\end{ExampleCode}
\end{Examples}
\inputencoding{utf8}
\HeaderA{Telomeresfinder}{Identify sequences containing canonical telomeric repeats}{Telomeresfinder}
%
\begin{Description}
This function searches for canonical telomeric repeats in FASTA files
and returns only the sequences that contain a minimum number of consecutive
repeat units. It also automatically considers the reverse, complement,
and reverse-complement versions of the repeat.
\end{Description}
%
\begin{Usage}
\begin{verbatim}
Telomeresfinder(
  fasta_file,
  telomere_repeat = NULL,
  min_repeats = NULL,
  output_fasta = "telomere_hits.fasta"
)
\end{verbatim}
\end{Usage}
%
\begin{Arguments}
\begin{ldescription}
\item[\code{fasta\_file}] Path to the FASTA file.

\item[\code{telomere\_repeat}] Canonical telomeric repeat sequence
(example: `"TTAGGG"`).

\item[\code{min\_repeats}] Minimum number of consecutive repeats required
for the sequence to be considered as containing telomere repeats.

\item[\code{output\_fasta}] Optional path to the output FASTA file.
\end{ldescription}
\end{Arguments}
%
\begin{Value}
A `DNAStringSet` object containing only the sequences that meet the criteria.
\end{Value}
%
\begin{Examples}
\begin{ExampleCode}
## Not run: 
hits <- Telomeresfinder("reads.fasta", telomere_repeat = "TTAGGG", min_repeats = 3)

## End(Not run)

\end{ExampleCode}
\end{Examples}
\printindex{}
\end{document}
